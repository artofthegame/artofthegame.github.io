\documentclass[12 pt]{article}
%\usepackage{geometry}
\usepackage[inner=1in,outer=1in,top=2.5cm,bottom=2.5cm]{geometry}
\pagestyle{empty}
\usepackage{lmodern}
\usepackage{graphicx}
\usepackage{fancyhdr, lastpage, bbding, pmboxdraw}
\usepackage[usenames,dvipsnames]{color}
\definecolor{darkblue}{rgb}{0,0,.6}
\usepackage{enumerate}
\usepackage[colorlinks,pagebackref,pdfusetitle,urlcolor=darkblue,citecolor=darkblue,linkcolor=darkred,bookmarksnumbered,plainpages=false]{hyperref}
\renewcommand{\thefootnote}{\fnsymbol{footnote}}

\pagestyle{fancyplain}
\fancyhf{}
\lhead{ \fancyplain{}{Course Name} }
%\chead{ \fancyplain{}{} }
\rhead{ \fancyplain{}{\today} }

\thispagestyle{plain}

%%%%%%%%%%%% LISTING %%%
\usepackage{listings}
\usepackage{caption}
\DeclareCaptionFont{white}{\color{white}}
\DeclareCaptionFormat{listing}{\colorbox{gray}{\parbox{\textwidth}{#1#2#3}}}
\captionsetup[lstlisting]{format=listing,labelfont=white,textfont=white}
\usepackage{verbatim} % used to display code
\usepackage{fancyvrb}
\usepackage{acronym}
\usepackage{amsthm}
\VerbatimFootnotes % Required, otherwise verbatim does not work in footnotes!
\usepackage[utf8]{inputenc}
\usepackage{helvet}




\definecolor{OliveGreen}{cmyk}{0.64,0,0.95,0.40}
\definecolor{CadetBlue}{cmyk}{0.62,0.57,0.23,0}
\definecolor{lightlightgray}{gray}{0.93}



\lstset{
%language=bash,                          % Code langugage
basicstyle=\ttfamily,                   % Code font, Examples: \footnotesize, \ttfamily
keywordstyle=\color{OliveGreen},        % Keywords font ('*' = uppercase)
commentstyle=\color{gray},              % Comments font
numbers=left,                           % Line nums position
numberstyle=\tiny,                      % Line-numbers fonts
stepnumber=1,                           % Step between two line-numbers
numbersep=5pt,                          % How far are line-numbers from code
backgroundcolor=\color{lightlightgray}, % Choose background color
frame=none,                             % A frame around the code
tabsize=2,                              % Default tab size
captionpos=t,                           % Caption-position = bottom
breaklines=true,                        % Automatic line breaking?
breakatwhitespace=false,                % Automatic breaks only at whitespace?
showspaces=false,                       % Dont make spaces visible
showtabs=false,                         % Dont make tabls visible
columns=flexible,                       % Column format
morekeywords={__global__, __device__},  % CUDA specific keywords
}

%%%%%%%%%%%%%%%%%%%%%%%%%%%%%%%%%%%%
\begin{document}
\begin{center}
{\Large \textsc{Final Project Guidelines}}
\end{center}
%\date{September 26, 2014}
\renewcommand{\arraystretch}{2}
\noindent\textbf{Final Essay (15\%)}
\begin{enumerate}[(i)]
\item Choose/Analyze a strategy for one of the games we have played
\begin{itemize}
\item Egyptian Rat Slap, Kremps,Texas HoldEm, Hearts, Coup, Settlers of Catan, or Avalon
\item The strategy does not necessarily need to be something you play tested during this semester. Just some successful strategy you observed, whether implemented by you or someone else.
\end{itemize}
\item Explore the strategy in your essay (perhaps answer these questions, or others)
\begin{itemize}
\item What game is your strategy for (pretty important?, so please make obvious)?
\item How do you implement it?
\item Why does it work?
\item What are some countermoves that another player can use to defend against your strategy?
\end{itemize}
\item Formatting \begin{itemize}
\item 1 page max, double spaced, 12-point Times New Roman, 1 inch margins
\item Name, Student ID, Title at top
\end{itemize}
\end{enumerate}
Example essays posted at @ https://artofthegame.github.io/FinalProject.html
\vskip 0.2 in
\noindent\textbf{Final Presentation (10\%)} 
\begin{enumerate}[(i)]
\item Choose/Analyze a strategy for one of the games we have played
\item Prepare a 4-5 minute presentation about the game discussing the following points:
\begin{enumerate}[(1)]
\item Rules
	\begin{itemize}
	\item You do not need to explain every single edge case rule, just the general idea of the game, enough to understand the main idea for the rest of presentation
	\end{itemize}
\item How it applies to this class (perhaps answer one of these, or others)
	\begin{itemize}
	\item Strategies?
	\item Number of players?
	\item Length of game?
	\end{itemize}
\item Note: Even if the game is not perfect in every aspect (e.g. longer than the 2 hour class time), still present on it if you find it interesting enough
\item Prepare for a few questions after
\item Any presentations exceeding 5 minutes will be ended
\item Sample presentation will be done in class on April 4, 2018
\end{enumerate}
\end{enumerate}
\end{document}

