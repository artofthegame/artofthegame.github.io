\documentclass[11.5pt]{article}

%\usepackage{geometry}
\usepackage[inner=1in,outer=1in,top=2.5cm,bottom=2.5cm]{geometry}
\pagestyle{empty}
\usepackage{lmodern}
\usepackage{graphicx}
\usepackage{fancyhdr, lastpage, bbding, pmboxdraw}
\usepackage[usenames,dvipsnames]{color}
\definecolor{darkblue}{rgb}{0,0,.6}
\usepackage[colorlinks,pagebackref,pdfusetitle,urlcolor=darkblue,citecolor=darkblue,linkcolor=darkred,bookmarksnumbered,plainpages=false]{hyperref}
\renewcommand{\thefootnote}{\fnsymbol{footnote}}

\pagestyle{fancyplain}
\fancyhf{}
\lhead{ \fancyplain{}{Course Name} }
%\chead{ \fancyplain{}{} }
\rhead{ \fancyplain{}{\today} }

\thispagestyle{plain}

%%%%%%%%%%%% LISTING %%%
\usepackage{listings}
\usepackage{caption}
\DeclareCaptionFont{white}{\color{white}}
\DeclareCaptionFormat{listing}{\colorbox{gray}{\parbox{\textwidth}{#1#2#3}}}
\captionsetup[lstlisting]{format=listing,labelfont=white,textfont=white}
\usepackage{verbatim} % used to display code
\usepackage{fancyvrb}
\usepackage{acronym}
\usepackage{amsthm}
\VerbatimFootnotes % Required, otherwise verbatim does not work in footnotes!
\usepackage[utf8]{inputenc}
\usepackage{enumitem}




\definecolor{OliveGreen}{cmyk}{0.64,0,0.95,0.40}
\definecolor{CadetBlue}{cmyk}{0.62,0.57,0.23,0}
\definecolor{lightlightgray}{gray}{0.93}



\lstset{
%language=bash,                          % Code langugage
basicstyle=\ttfamily,                   % Code font, Examples: \footnotesize, \ttfamily
keywordstyle=\color{OliveGreen},        % Keywords font ('*' = uppercase)
commentstyle=\color{gray},              % Comments font
numbers=left,                           % Line nums position
numberstyle=\tiny,                      % Line-numbers fonts
stepnumber=1,                           % Step between two line-numbers
numbersep=5pt,                          % How far are line-numbers from code
backgroundcolor=\color{lightlightgray}, % Choose background color
frame=none,                             % A frame around the code
tabsize=2,                              % Default tab size
captionpos=t,                           % Caption-position = bottom
breaklines=true,                        % Automatic line breaking?
breakatwhitespace=false,                % Automatic breaks only at whitespace?
showspaces=false,                       % Dont make spaces visible
showtabs=false,                         % Dont make tabls visible
columns=flexible,                       % Column format
morekeywords={__global__, __device__},  % CUDA specific keywords
}

%%%%%%%%%%%%%%%%%%%%%%%%%%%%%%%%%%%%
\begin{document}
\begin{center}
{\Large \textsc{Coup Rulebook}}
\end{center}
\begin{center}
By Taejin Hwang
\end{center}
%\date{September 26, 2014}
\renewcommand{\arraystretch}{2}

\vskip .1in
\noindent\textbf{Number of Players:} 3-6 
\vskip 0.1 in
\noindent\textbf{Objective:} Be the last player standing with influence cards left.
\vskip 0.1in
\noindent\textbf{Materials:} Deck of influence cards, bank of coins.
\vskip 0.1in
\noindent\textbf{Setup:} Shuffle the cards and deal two to each player. Players should look at their cards but keep them
hidden from everyone else. Each player takes two coins from the bank as their starting wealth.
\vskip 0.1in
\noindent\textbf{Cards:} There are five different characters in the influence deck (three copies of each character). Each
of the five characters has one or more special abilities. 
\begin{enumerate}
\setlength{\itemsep}{3pt}
\item The Duke takes taxes and Blocks Foreign Aid.
\item The Assassin forces one player to give up an Influence card at the cost of 3 coins.
\item The Captain steals two coins from another player and blocks stealing attempts. 
\item The Ambassador lets you swap or appear to swap your Influence cards with two new ones from the deck and blocks stealing attempts.
\item The Contessa blocks assassination attempts.
\end{enumerate}
\noindent \textbf{Play:} Starting with the player to the left of the dealer and going clockwise, players take turns
performing one of the available actions. 
\begin{enumerate}
\setlength{\itemsep}{3pt}
\item Income: Take one coin from the bank. This cannot be challenged nor blocked.
\item Foreign Aid: Take two coins from the bank. This cannot be challenged but it can be blocked by
the Duke.
\item Coup: Costs seven coins. Causes a player to give up an Influence card. Cannot be challenged nor
blocked. If you start your turn with 10 or more coins, you must take this action.
\item Taxes (the Duke): Take three coins from the bank. Can be challenged.
\item Assassinate (the Assassin): Costs three coins. Force one player to give up an Influence card of
their choice. Can be challenged. Can be blocked by the Contessa.
\item Steal (the Captain): Take two coins from another player. Can be challenged. Can be blocked by
another Captain or an Ambassador.
\item Exchange (the Ambassador): Draw two Influence cards from the deck, look at them and mix them with your current Influence card(s). Place two cards back in the deck and shuffle the deck. Can be challenged. cannot be Blocked.
\end{enumerate}
\noindent \textbf{Blocking:} If a player takes an action that can be blocked, other players may block it by claiming to have the proper character. The acting player cannot perform the the action and any other action this turn. However, if the acting player chooses to challenge the blocking player and wins the challenge, the action goes through.
\vskip 0.12cm
\noindent \textbf{Challenge:}  Whenever a player takes their action or another player blocks someone's action, the block or action can be challenged. For challenging a block, any player can challenge the block even if they were not involved. The challenged player must now prove they have the card for the action/block or forfeit the challenge. If they have the corresponding character, they reveal it, the action/block goes through, and they place the revealed card back into the deck. They then shuffle the deck and draw a new card. The challenging player has lost the challenge. If they do NOT have the proper character, then they have lost the challenge.
\vskip 0.12cm
\noindent \textbf{Losing a Challenge:} Any player who loses a challenge must turn one of their Influence cards face up
for all to see. If that is their last Influence card, they are out of the game. 
\vskip 0.12cm
\noindent \textbf{Losing Influence:} Any time a player loses an Influence card, THEY choose which of their cards to reveal. These cards are not returned to the deck. 
\end{document}