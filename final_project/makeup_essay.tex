\documentclass[11.5pt]{article}

%\usepackage{geometry}
\usepackage[inner=1in,outer=1in,top=2.5cm,bottom=2.5cm]{geometry}
\pagestyle{empty}
\usepackage{lmodern}
\usepackage{graphicx}
\usepackage{fancyhdr, lastpage, bbding, pmboxdraw}
\usepackage[usenames,dvipsnames]{color}
\definecolor{darkblue}{rgb}{0,0,.6}
\usepackage[colorlinks,pagebackref,pdfusetitle,urlcolor=darkblue,citecolor=darkblue,linkcolor=darkred,bookmarksnumbered,plainpages=false]{hyperref}
\renewcommand{\thefootnote}{\fnsymbol{footnote}}

\pagestyle{fancyplain}
\fancyhf{}
\lhead{ \fancyplain{}{The Art of the Game} }
%\chead{ \fancyplain{}{} }
\rhead{ \fancyplain{}{\today} }

\thispagestyle{plain}

%%%%%%%%%%%% LISTING %%%
\usepackage{listings}
\usepackage{caption}
\DeclareCaptionFont{white}{\color{white}}
\DeclareCaptionFormat{listing}{\colorbox{gray}{\parbox{\textwidth}{#1#2#3}}}
\captionsetup[lstlisting]{format=listing,labelfont=white,textfont=white}
\usepackage{verbatim} % used to display code
\usepackage{fancyvrb}
\usepackage{acronym}
\usepackage{amsthm}
\VerbatimFootnotes % Required, otherwise verbatim does not work in footnotes!
\usepackage[utf8]{inputenc}
\usepackage{enumerate}




\definecolor{OliveGreen}{cmyk}{0.64,0,0.95,0.40}
\definecolor{CadetBlue}{cmyk}{0.62,0.57,0.23,0}
\definecolor{lightlightgray}{gray}{0.93}



\lstset{
%language=bash,                          % Code langugage
basicstyle=\ttfamily,                   % Code font, Examples: \footnotesize, \ttfamily
keywordstyle=\color{OliveGreen},        % Keywords font ('*' = uppercase)
commentstyle=\color{gray},              % Comments font
numbers=left,                           % Line nums position
numberstyle=\tiny,                      % Line-numbers fonts
stepnumber=1,                           % Step between two line-numbers
numbersep=5pt,                          % How far are line-numbers from code
backgroundcolor=\color{lightlightgray}, % Choose background color
frame=none,                             % A frame around the code
tabsize=2,                              % Default tab size
captionpos=t,                           % Caption-position = bottom
breaklines=true,                        % Automatic line breaking?
breakatwhitespace=false,                % Automatic breaks only at whitespace?
showspaces=false,                       % Dont make spaces visible
showtabs=false,                         % Dont make tabls visible
columns=flexible,                       % Column format
morekeywords={__global__, __device__},  % CUDA specific keywords
}

%%%%%%%%%%%%%%%%%%%%%%%%%%%%%%%%%%%%
\begin{document}
\begin{center}
{\Large \textsc{The Art of the Game}}
\end{center}
\begin{center}
MAKEUP ESSAY
\end{center}
%\date{September 26, 2014}
\renewcommand{\arraystretch}{2}
\vskip 0.1in
\noindent\textbf{Deliverables:} The guidelines and ex
\vskip .1in
\noindent\textbf{Grading:} The entire project as a whole will be worth 25\% of your grade in the class. The final essay will be worth 60\% and the remaining 40\% will be a proposal of a new rule for one of the games we have played.
\vskip 0.3in
\noindent\textbf{Part 1: Final Essay} 
\begin{enumerate}
\setlength{\itemsep}{3pt}
\item Choose and analyze a strategy for one of the games we've played:
	\begin{enumerate}[(i)]
	\item Your options are: Kemps, Hearts, Catan, Coup, Codenames, Avalon, and Secret Hitler
	\item The strategy does not necessarily need to be something you play tested during this semester. Just some successful strategy you observed, whether implemented by you or someone else.
	\end{enumerate}
\item Explore the strategy in your essay (perhaps answer these questions, or others)
	\begin{enumerate}[(i)]
	\item What game is your strategy for? (Please make this obvious)
	\item How is it implemented?
	\item When does it work and why?
	\item What are some drawbacks of this strategy? How can it be countered?
	\end{enumerate}
\item Formatting
	\begin{enumerate}[(i)]
	\item 1 page max, double spaced, 12pt Times New Roman, 1 inch margins
	\item Please keep your essay to one page, keep your points concise.
	\item Name, Student ID, Title at the top
	\item Sample Essay: 
	\end{enumerate}
\item Deadline / How to turn in
	\begin{enumerate}[(i)]
	\item The final essay will be due on the last day of class May 1st.
	\item Please email your essay to calaotg@gmail.com.
	\item Please note that the Essay and Strategy Proposal have DIFFERENT deadlines.
	\end{enumerate}
\end{enumerate}

\newpage
\noindent\textbf{Part 2: New Rule Proposal} 
\begin{enumerate}
\setlength{\itemsep}{3pt}
\item In this part you will come up with a new rule / character / modification to a game we've played.
	\begin{enumerate}[(i)]
	\item Your options are the same: Kemps, Hearts, Catan, Coup, Codenames, Avalon, and Secret Hitler
	\item Your modification is something that should be simple to implement. The majority of the rules should still stay the same.
	\end{enumerate}
\item Explore the strategy in your project (perhaps answer these questions, or others)
	\begin{enumerate}[(i)]
	\item What game is your strategy for? (Please make this obvious)
	\item What is your new rule / strategy / modification?
	\item How is it implemented?
	\item How does this change gameplay?
	\item Do you think your change is beneficial to the game? If so, or if not, why?
	\item Are there strategies currently in the game that will not work given your change?
	\end{enumerate}
\item Formatting
	\begin{enumerate}[(i)]
	\item For this part, the formatting is up to you. You may make a video, type up a report, or do some form of presentation.
		\begin{enumerate}[(a)]
		\item If you do a video, please keep it to a maximum of 5 minutes
		\item If you write up a report, please keep it to a minimum of 1 page, maximum of 5 pages.
		\end{enumerate}
	\item You may include any drawings, pictures, or whatever you feel is relevant to convey your ideas.
	\item Name, Student ID, Title at the top
	\item Sample will be on the course website
	\end{enumerate}
\item Final Day of Class
	\begin{enumerate}[(i)]
	\item For this portion of the project, we will select the top strategies and actually try them out in class.
	\item This obviously means this portion of the project will actually be due before the final day of class.
	\end{enumerate}
\item Deadline / How to turn in
	\begin{enumerate}[(i)]
	\item This portion of the final project will be due Monday April 29th at 11:59 p.m.
	\item Please email your project to calaotg@gmail.com
	\item If you're unsure about how to submit your project, please ask.
	\end{enumerate}
\end{enumerate}
\end{document}