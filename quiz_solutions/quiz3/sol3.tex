\documentclass[11.5pt]{article}

%\usepackage{geometry}
\usepackage[inner=0.9in,outer=0.9in,top=2.5cm,bottom=2.5cm]{geometry}
\pagestyle{empty}
\usepackage{lmodern}
\usepackage{graphicx}
\usepackage{fancyhdr, lastpage, bbding, pmboxdraw}
\usepackage[usenames,dvipsnames]{color}
\definecolor{darkblue}{rgb}{0,0,.6}
\usepackage[colorlinks,pagebackref,pdfusetitle,urlcolor=darkblue,citecolor=darkblue,linkcolor=darkred,bookmarksnumbered,plainpages=false]{hyperref}
\renewcommand{\thefootnote}{\fnsymbol{footnote}}

\pagestyle{fancyplain}
\fancyhf{}
\lhead{ \fancyplain{}{The Art of the Game} }
%\chead{ \fancyplain{}{} }
\rhead{ \fancyplain{}{\today} }

\thispagestyle{plain}

%%%%%%%%%%%% LISTING %%%
\usepackage{listings}
\usepackage{caption}
\DeclareCaptionFont{white}{\color{white}}
\DeclareCaptionFormat{listing}{\colorbox{gray}{\parbox{\textwidth}{#1#2#3}}}
\captionsetup[lstlisting]{format=listing,labelfont=white,textfont=white}
\usepackage{verbatim} % used to display code
\usepackage{fancyvrb}
\usepackage{acronym}
\usepackage{amsthm}
\VerbatimFootnotes % Required, otherwise verbatim does not work in footnotes!
\usepackage[utf8]{inputenc}
\usepackage{enumitem}




\definecolor{OliveGreen}{cmyk}{0.64,0,0.95,0.40}
\definecolor{CadetBlue}{cmyk}{0.62,0.57,0.23,0}
\definecolor{lightlightgray}{gray}{0.93}



\lstset{
%language=bash,                          % Code langugage
basicstyle=\ttfamily,                   % Code font, Examples: \footnotesize, \ttfamily
keywordstyle=\color{OliveGreen},        % Keywords font ('*' = uppercase)
commentstyle=\color{gray},              % Comments font
numbers=left,                           % Line nums position
numberstyle=\tiny,                      % Line-numbers fonts
stepnumber=1,                           % Step between two line-numbers
numbersep=5pt,                          % How far are line-numbers from code
backgroundcolor=\color{lightlightgray}, % Choose background color
frame=none,                             % A frame around the code
tabsize=2,                              % Default tab size
captionpos=t,                           % Caption-position = bottom
breaklines=true,                        % Automatic line breaking?
breakatwhitespace=false,                % Automatic breaks only at whitespace?
showspaces=false,                       % Dont make spaces visible
showtabs=false,                         % Dont make tabls visible
columns=flexible,                       % Column format
morekeywords={__global__, __device__},  % CUDA specific keywords
}

%%%%%%%%%%%%%%%%%%%%%%%%%%%%%%%%%%%%
\begin{document}
\begin{center}
{\Large \textsc{The Art of the Game}}
\end{center}
\begin{center}
Quiz 1 Solutions
\end{center}
%\date{September 26, 2014}
\renewcommand{\arraystretch}{2}

\vskip .1in
\noindent 1: Are you allowed to play a development card the turn you buy it?
\vskip 0.1 in
\indent \textbf{Answer: Yes, but only if the development card is a Victory Point and it brings me to 10 points.} 

\vskip 0.16in

\vskip .1in
\noindent 2: Are you allowed to play a development card before you roll the die??
\vskip 0.1 in
\indent \textbf{Yes, you are allowed to play cards such as the knight before you roll.}

\vskip 0.16in

\vskip .1in
\noindent 3: What does rolling a 7 allow you to do (mark all that apply)?
\vskip 0.1 in
\indent \textbf{There is no space on the board with a 7 on it. When a 7 is rolled, everyone who has strictly more than 7 (8 or more) cards must discard half of their cards rounding down. In addition to this, the player that rolled a 7, must move the robber to a new location and he or she is also allowed to steal a resource from any player who owns a settlement adjacent to the tile the robber was moved to. }

\vskip 0.16in

\vskip .1in
\noindent 4: What do you need to build a settlement (mark all that apply)?
\vskip 0.1 in
\indent \textbf{Wood, Brick, Wheat, Sheep}

\vskip 0.16in

\vskip .1in
\noindent 5: For the following image, where is an acceptable location for the red player to build a settlement?
\vskip 0.1 in
\indent \textbf{The rules of Catan say that a player is not allowed to build a settlement on a spot adjacent to another settlement. Therefore location B will be the appropriate spot.}

\vskip 0.16in

\vskip .1in
\noindent 6: Alice has currently played the Knight card 4 times and Bob has played it 3 times. It is Bob's turn and he plays one of his Knight cards. Who owns the Largest Army card now?
\vskip 0.1 in
\indent \textbf{Only Alice would still own the Largest Army card. Remember that in order for a person to take the Largest Army card, he or she must be the first to reach 3 knights or must pass whoever has the most number of knights currently.}

\vskip 0.16in

\noindent 7: Extra Credit (2 points): Extra Credit (2 points): There are three players in the game. Alice, Bob, and Chris. Alice has 8 points, Bob has 9 points, and Chris has 9 points. At the moment, nobody has any victory points. Alice's longest road at the moment is 6, Bob's longest road is 4 and Chris's longest road is 9. It is currently Bob's turn and he builds a road and then a settlement that interferes at the middle of Chris's longest road. This means Alice now has the longest road of 6! Bob then buys a development card and it happens to be a Victory Point. Who is the winner of the game and why?
\vskip 0.1 in
\indent \textbf{There are many rules at play here. We first realize that Chris has a total of 9 points and two of them are from the longest road card. Now Bob builds a road and a settlement bringing him up to 9 points. Since he broke Chris's longest road, Alice now has the longest road and she takes Chris's largest army card bringing her up to 10 points. Bob then buys a development card and it happens to be a victory point so he is now at 10 points. Since the victory point has brought him up to 10 points, he is allowed to play the victory point and declare himself the winner. Alice, although she reached 10 points does not win the game since it is not her turn and Bob reached 10 points ON his turn first.}
\end{document}