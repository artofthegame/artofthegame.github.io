\documentclass[11.5pt]{article}

%\usepackage{geometry}
\usepackage[inner=0.9in,outer=0.9in,top=2.5cm,bottom=2.5cm]{geometry}
\pagestyle{empty}
\usepackage{lmodern}
\usepackage{graphicx}
\usepackage{fancyhdr, lastpage, bbding, pmboxdraw}
\usepackage[usenames,dvipsnames]{color}
\definecolor{darkblue}{rgb}{0,0,.6}
\usepackage[colorlinks,pagebackref,pdfusetitle,urlcolor=darkblue,citecolor=darkblue,linkcolor=darkred,bookmarksnumbered,plainpages=false]{hyperref}
\renewcommand{\thefootnote}{\fnsymbol{footnote}}

\pagestyle{fancyplain}
\fancyhf{}
\lhead{ \fancyplain{}{The Art of the Game} }
%\chead{ \fancyplain{}{} }
\rhead{ \fancyplain{}{\today} }

\thispagestyle{plain}

%%%%%%%%%%%% LISTING %%%
\usepackage{listings}
\usepackage{caption}
\DeclareCaptionFont{white}{\color{white}}
\DeclareCaptionFormat{listing}{\colorbox{gray}{\parbox{\textwidth}{#1#2#3}}}
\captionsetup[lstlisting]{format=listing,labelfont=white,textfont=white}
\usepackage{verbatim} % used to display code
\usepackage{fancyvrb}
\usepackage{acronym}
\usepackage{amsthm}
\VerbatimFootnotes % Required, otherwise verbatim does not work in footnotes!
\usepackage[utf8]{inputenc}
\usepackage{enumitem}




\definecolor{OliveGreen}{cmyk}{0.64,0,0.95,0.40}
\definecolor{CadetBlue}{cmyk}{0.62,0.57,0.23,0}
\definecolor{lightlightgray}{gray}{0.93}



\lstset{
%language=bash,                          % Code langugage
basicstyle=\ttfamily,                   % Code font, Examples: \footnotesize, \ttfamily
keywordstyle=\color{OliveGreen},        % Keywords font ('*' = uppercase)
commentstyle=\color{gray},              % Comments font
numbers=left,                           % Line nums position
numberstyle=\tiny,                      % Line-numbers fonts
stepnumber=1,                           % Step between two line-numbers
numbersep=5pt,                          % How far are line-numbers from code
backgroundcolor=\color{lightlightgray}, % Choose background color
frame=none,                             % A frame around the code
tabsize=2,                              % Default tab size
captionpos=t,                           % Caption-position = bottom
breaklines=true,                        % Automatic line breaking?
breakatwhitespace=false,                % Automatic breaks only at whitespace?
showspaces=false,                       % Dont make spaces visible
showtabs=false,                         % Dont make tabls visible
columns=flexible,                       % Column format
morekeywords={__global__, __device__},  % CUDA specific keywords
}

%%%%%%%%%%%%%%%%%%%%%%%%%%%%%%%%%%%%
\begin{document}
\begin{center}
{\Large \textsc{The Art of the Game}}
\end{center}
\begin{center}
Spring 2019
\end{center}
%\date{September 26, 2014}
\renewcommand{\arraystretch}{2}

\vskip .1in
\noindent\textbf{Facilitators:} George Nacouzi, Taejin Hwang 
\vskip 0.1 in
\noindent\textbf{Time and Location:} Wed 6:30--8:30, 81 Evans
\vskip 0.1in
\noindent\textbf{Email:}  calaotg@gmail.com
\vspace*{.15in}

\noindent \textbf{Course Description:} \vskip 0.1cm
\noindent This course is designed for anyone with an interest and love for strategy games. It is made for people with the ambition to build their game skills regardless of prior experience. Students who are new to the game have the opportunity to learn the basics and rules of the game. Students with prior experience have the opportunity to refine their skills. The course covers a variety of games such those of card, board, and social deduction.
\vspace*{0.15 in}

\noindent \textbf{Course Expectations:} \vskip 0.1cm
\noindent In the first 10 minutes of class, we will have a quiz testing whether or not you have done the assigned readings. The next 15-20 minutes will be a brief review of game rules and strategies. The remaining time up until the last 15 will be spent playing the actual games. The last 15 minutes will be spent analyzing individual play from the games.
\vspace*{0.15in}

\noindent \textbf{Course Readings:} Readings will be on the course webpage. 

\vspace*{0.15in}
\noindent \textbf{Coures Webpage:} \url{https://artofthegame.github.io}
 
\vspace*{.25in}
\noindent\textbf{\large Grading Distribuiton:} 
\begin{description}
\item \textbf{Attendance \& Participation:} 55\% 
\end{description}

\begin{itemize}
\item Please contact prior to class if you will be absent so we can plan ahead.
\item Each absence after will result in a 5\% penalty.
\item More than 4 absences will result in failure of the course.
\end{itemize}

\begin{description}
\item \textbf{Quizzes:} 20\% 
\end{description}

\begin{itemize}
\item Please come to class with a basic understanding of the rules/strategy of a
particular game. 
\item Weekly quizzes will be assigned to ensure readings were complete (Course
Readings are mandatory, further readings are heavily suggested.)
\end{itemize}

\begin{description}
\item \textbf{Essays:} 15\% 
\end{description}

\begin{itemize}
\item Students are expected to write up a final essay analyzing a game of their choice and a specific strategy. Another essay can be written at any point in the course to make up the points for an absence and quiz.
\end{itemize}

\begin{description}
\item \textbf{Final Presentation:} 10\% 
The scope for the final project is undecided at the moment.
\end{description}

\begin{itemize}
\item In-class presentation of a game you would like to see in a future offering of the course.
\end{itemize}
\newpage

\noindent\textbf{\large Schedule}: \vspace*{0.05cm}
\flushleft
\begin{tabular}{ |p{0.8cm}||p{3cm}|p{7cm}|p{4cm}|  }
 \hline
 Week & Date & Topic & Homework \\
 \hline
1  & 2/13/18 & Intro, Kemps & None \\ \hline
2 &  2/20/18  & Hearts & Instructions / Videos \\ \hline
3 & 2/27/18 & Spades & Instructions \\ \hline
4 & 3/6/18  & Rummy & Instructions / Strategy \\ \hline
5 & 3/13/18 & Settlers of Catan & Instructions \\ \hline
6 & 3/20/18 & Coup: The Dystopian Universe & Instructions / Videos \\ \hline
7 & 4/3/18 & The Resistance: Avalon & Instructions / Videos \\ \hline
8 & 4/17/18 & The Resistance: Avalon & TBD \\ \hline
9 & 4/24/18 & Favorite Game & None \\ \hline
10 & 5/1/18 & In-class Final Presentation & None \\
\hline
\end{tabular}
\vskip 0.5 cm
\noindent\textbf{UC Berkeley Honor Code:} \\
As a member of the UC Berkeley community, I act with honesty, integrity, and respect for others.
\end{document}